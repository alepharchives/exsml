\documentclass[a4paper, oneside, 10pt, final]{memoir}
%% This file is my default inclusion of packages I need when writing
%% things in LaTeX. The list here is pretty comprehensive, but sets up
%% the main stuff I use in all projects I start on.

%% This part sets up the font basics. I like mathpazo a lot. It sets
%% up Palatino as a clone with pazo math. It is much more beautiful
%% than many other fonts you can choose from. I also like some of the
%% mathdesign variants, in particular [utopia,osf]{mathdesign} with no
%% linespread. We could make it a setting here.
%%
%% Also, add microtype. We can't really live without it.
\usepackage{microtype}
\usepackage[T1]{fontenc}
\usepackage{lmodern}
%\usepackage[osf]{mathpazo}
%\linespread{1.05}
%\usepackage[sc,osf,utopia]{mathdesign}

%% AMS gives us the basic math-support
\usepackage{amsmath}

%% AMS Theorem support. We set up a good list of theorems and make
%% sure that they are numbered appropriately in the order of main
%% theorems. I hate when a document contains too many counters so this
%% keeps the number of live counters down to a minimum.
\usepackage{amsthm}
\theoremstyle{plain}
\newtheorem{axm}{Axiom}
\newtheorem{thm}{Theorem}
\newtheorem{lem}[thm]{Lemma}
\newtheorem{prop}[thm]{Proposition}
\newtheorem*{cor}{Corollary}

\theoremstyle{definition}
\newtheorem{defn}[thm]{Definition}
\newtheorem{conj}[thm]{Conjecture}
\newtheorem{exmp}[thm]{Example}
\theoremstyle{remark}
\newtheorem*{rem}{Remark}
\newtheorem*{note}{Note}
\newtheorem{case}{Case}


%% Hyperref should be last
\usepackage{hyperref}

\chapterstyle{culver}
\usepackage{fixme}
\usepackage{coqdoc}
\usepackage[english]{babel}
%\usepackage[utopia]{mathdesign}

\usepackage[osf,sc]{mathpazo}
\linespread{1.05}
%\usepackage{fourier}


\usepackage{semantic}
\author{Jesper Louis
  Andersen\\jesper.louis.andersen@gmail.com\\140280-2029}
\title{Lambda}
\date{\today}

\newlength{\drop}
\newcommand*{\titleM}{\begingroup% Misericords, T&H p 153
  \drop = 0.08\textheight
  \centering
  {\Huge\bfseries Lambda}\\[\baselineskip]
  {\scshape IR of exsml}\\[\baselineskip]
  {\scshape by}\\[\baselineskip]
  {\large\scshape Jesper Louis Andersen\\jesper.louis.andersen@gmail.com}\par
  \endgroup}

\bibliographystyle{plain}

\begin{document}
\newcommand{\janusz}{$\mathrm{JANUS}_0$}
\newcommand{\januso}{$\mathrm{JANUS}_1$}
\newcommand{\lift}[1]{\lfloor #1 \rfloor}
\newcommand{\NN}{\mathbb{N}}
\newcommand{\ZZ}{\mathbb{Z}}
\titleM{}
\begin{abstract}
\end{abstract}
\listoffixmes
%\tableofcontents
\chapter{Introduction}
This document describes the Lambda-language which is the internal
representation of Moscow ML and ExSML. It is based upon a similar
language in the Ocaml system, although a number of changes has been
made to accomodate SML-specifics.

\chapter{Lambda}

Lambda is an extended lambda-calculus, much like most functional
programming languages use as a basis. By extended, we means that the
language has the lambda-calculus as its base but has been augmented
with additional constructs making certain operations more viable. As
an example, you can define Peano-arithmetic in the pure lambda
calculus -- but this construction is hardly efficient. In Lambda,
integers are provided as a primitive, yielding fast arithmetic. Also,
it is easier to map fast arithmetic down to the underlying machine.

To circumvent problems with alpha-conversion in the traditional
Lambda-calculus, the Lambda language uses De-Bruijn indexing all
over. With De-Bruijn indexing, machine manipulation is easy although
it is harder to read for a human being.

Lambda also contains while-loops and sequencing effectively making
the language into an ``algorithmic functional language''. Contrast with
a pure functional language in which there are no imperative constructs
at all.

First, we introduce some data types used in the Lambda language:

\section{Special Constructors}

To facilitate the inclusion of integers, reals, strings, words and
characters in the Lambda language, we use the concept of a
\emph{Special Constructor} (SCon). An SCon is a constructor in the
traditional sense from a datatype, but it is special in the sense it
is used for primitive values. In the backend a Special constructor
is encoded as a \texttt{value}.

Note that the Lambda definition uses an \texttt{StructConstant} as the
constant container rather than an SCon. This is also the reason SCons
are often called Atoms. We will cover StructConstants later.

\section{BlockTags}

A BlockTag is really a pair of a tag number and a span, embedded in a
\texttt{CONtag} constructor. The tag number is used as a discriminator
so different blocktags can be distinguished from each other. The span
records how many different tag numbers there are. For example, the
values ``false'' and ``true'' are encoded as $(0,2)$ and $(1,2)$
respectively.

This information is utilized by the pattern match compiler. It knows
when it has outruled the possibility of a match on $span-1$ then it
\emph{must} be the final value. Note that the tag number is not unique
over all different datatypes. We know, due to static type checking
that different tag numbers can not possibly intermingle.

\section{Syntax}




\end{document}
