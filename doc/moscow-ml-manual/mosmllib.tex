% mosmllib.tex v. 2.00.0 Copyright (C) Peter Sestoft 2000, 2000-06-01
%
% You may edit for lay-out, or leave out irrelevant sections (if
% such omissions are marked somehow), but you may not redistribute the
% sources.  The authors' names and the Moscow ML URL must be left in place.

\documentclass[fleqn,twoside,a4paper]{article}
\usepackage{geometry}

\usepackage{isolatin1,program,pslatex}
\usepackage[T1]{fontenc}

\makeindex
\pagestyle{headings}
\thispagestyle{empty}

\begin{document}

\begin{center}

  \vspace*{0cm}

{\huge\bf Moscow ML Library Documentation}\\[0.5cm]

{Version 2.00 of June 2000}\\[0.5cm]

Sergei Romanenko, Russian Academy of Sciences, Moscow, Russia\\
Claudio Russo, Cambridge University, Cambridge, United Kingdom\\
Peter Sestoft, Royal Veterinary and Agricultural University, 
Copenhagen, Denmark

% \begin{tabular}{ccc}
% \large Sergei Romanenko && \large Peter Sestoft\\[0.1cm]
% Russian Academy of Sciences && Royal Veterinary and Agricultural University\\
% Moscow, Russia && Copenhagen, Denmark
% \end{tabular}
\end{center}

\vspace{1cm}

\subsection*{This document}

This manual describes the Moscow ML library, which includes parts of
the SML Basis Library and several extensions.  The manual has been
generated automatically from the commented signature files.


\subsection*{Alternative formats of this document}

\subsubsection*{Hypertext on the World-Wide Web}
  
The manual is available at
\verb$http://www.dina.kvl.dk/~sestoft/mosmllib/$ for online browsing.


\subsubsection*{Hypertext in the Moscow ML distribution}
  
The manual is available for offline browsing at
\verb$mosml/doc/mosmllib/index.html$ in the distribution.

\subsubsection*{On-line help in the Moscow ML interactive system}

The manual is available also in interactive {\tt mosml} sessions.
Type {\tt help "lib";} for an overview of built-in function libraries.
Type {\tt help "fromstring";} for help on a particular identifier,
such as {\tt fromString}.  This will produce a menu of all library
structures which contain the identifier {\tt fromstring} (disregarding
the lowercase/uppercase distinction):

{\small\begin{verbatim}
    --------------------------------
    |   1 | val  Bool.fromString   |
    |   2 | val  Char.fromString   |
    |   3 | val  Date.fromString   |
    |   4 | val  Int.fromString    |
    |   5 | val  Path.fromString   |
    |   6 | val  Real.fromString   |
    |   7 | val  String.fromString |
    |   8 | val  Time.fromString   |
    |   9 | val  Word.fromString   |
    |  10 | val  Word8.fromString  |
    --------------------------------
\end{verbatim}}

\noindent Choosing a number from this menu will invoke the help
browser on the desired structure, e.g.\ {\tt Int}.  

\vfill

\begin{center}
\begin{tabular}{|c|}\hline
\rule[-0.4cm]{0cm}{1cm}The Moscow ML home page is\ \
    \verb$http://www.dina.kvl.dk/~sestoft/mosml.html$\\\hline
\end{tabular}
\end{center}

\newpage

\setcounter{page}{1}

\twocolumn
{\setlength{\parindent}{0cm}
\renewcommand{\contentsline}[3]{#2\hfill #3\hspace*{3em}\newline}
\tableofcontents
}
\onecolumn

\newpage 

\input{texsigsigs.tex}

\newpage

\markboth{\MakeUppercase{Index}}{\MakeUppercase{Index}}
\addcontentsline{toc}{section}{Index}
% mosmllib.tex v. 2.00.0 Copyright (C) Peter Sestoft 2000, 2000-06-01
%
% You may edit for lay-out, or leave out irrelevant sections (if
% such omissions are marked somehow), but you may not redistribute the
% sources.  The authors' names and the Moscow ML URL must be left in place.

\documentclass[fleqn,twoside,a4paper]{article}
\usepackage{geometry}

\usepackage{isolatin1,program,pslatex}
\usepackage[T1]{fontenc}

\makeindex
\pagestyle{headings}
\thispagestyle{empty}

\begin{document}

\begin{center}

  \vspace*{0cm}

{\huge\bf Moscow ML Library Documentation}\\[0.5cm]

{Version 2.00 of June 2000}\\[0.5cm]

Sergei Romanenko, Russian Academy of Sciences, Moscow, Russia\\
Claudio Russo, Cambridge University, Cambridge, United Kingdom\\
Peter Sestoft, Royal Veterinary and Agricultural University, 
Copenhagen, Denmark

% \begin{tabular}{ccc}
% \large Sergei Romanenko && \large Peter Sestoft\\[0.1cm]
% Russian Academy of Sciences && Royal Veterinary and Agricultural University\\
% Moscow, Russia && Copenhagen, Denmark
% \end{tabular}
\end{center}

\vspace{1cm}

\subsection*{This document}

This manual describes the Moscow ML library, which includes parts of
the SML Basis Library and several extensions.  The manual has been
generated automatically from the commented signature files.


\subsection*{Alternative formats of this document}

\subsubsection*{Hypertext on the World-Wide Web}
  
The manual is available at
\verb$http://www.dina.kvl.dk/~sestoft/mosmllib/$ for online browsing.


\subsubsection*{Hypertext in the Moscow ML distribution}
  
The manual is available for offline browsing at
\verb$mosml/doc/mosmllib/index.html$ in the distribution.

\subsubsection*{On-line help in the Moscow ML interactive system}

The manual is available also in interactive {\tt mosml} sessions.
Type {\tt help "lib";} for an overview of built-in function libraries.
Type {\tt help "fromstring";} for help on a particular identifier,
such as {\tt fromString}.  This will produce a menu of all library
structures which contain the identifier {\tt fromstring} (disregarding
the lowercase/uppercase distinction):

{\small\begin{verbatim}
    --------------------------------
    |   1 | val  Bool.fromString   |
    |   2 | val  Char.fromString   |
    |   3 | val  Date.fromString   |
    |   4 | val  Int.fromString    |
    |   5 | val  Path.fromString   |
    |   6 | val  Real.fromString   |
    |   7 | val  String.fromString |
    |   8 | val  Time.fromString   |
    |   9 | val  Word.fromString   |
    |  10 | val  Word8.fromString  |
    --------------------------------
\end{verbatim}}

\noindent Choosing a number from this menu will invoke the help
browser on the desired structure, e.g.\ {\tt Int}.  

\vfill

\begin{center}
\begin{tabular}{|c|}\hline
\rule[-0.4cm]{0cm}{1cm}The Moscow ML home page is\ \
    \verb$http://www.dina.kvl.dk/~sestoft/mosml.html$\\\hline
\end{tabular}
\end{center}

\newpage

\setcounter{page}{1}

\twocolumn
{\setlength{\parindent}{0cm}
\renewcommand{\contentsline}[3]{#2\hfill #3\hspace*{3em}\newline}
\tableofcontents
}
\onecolumn

\newpage 

\input{texsigsigs.tex}

\newpage

\markboth{\MakeUppercase{Index}}{\MakeUppercase{Index}}
\addcontentsline{toc}{section}{Index}
% mosmllib.tex v. 2.00.0 Copyright (C) Peter Sestoft 2000, 2000-06-01
%
% You may edit for lay-out, or leave out irrelevant sections (if
% such omissions are marked somehow), but you may not redistribute the
% sources.  The authors' names and the Moscow ML URL must be left in place.

\documentclass[fleqn,twoside,a4paper]{article}
\usepackage{geometry}

\usepackage{isolatin1,program,pslatex}
\usepackage[T1]{fontenc}

\makeindex
\pagestyle{headings}
\thispagestyle{empty}

\begin{document}

\begin{center}

  \vspace*{0cm}

{\huge\bf Moscow ML Library Documentation}\\[0.5cm]

{Version 2.00 of June 2000}\\[0.5cm]

Sergei Romanenko, Russian Academy of Sciences, Moscow, Russia\\
Claudio Russo, Cambridge University, Cambridge, United Kingdom\\
Peter Sestoft, Royal Veterinary and Agricultural University, 
Copenhagen, Denmark

% \begin{tabular}{ccc}
% \large Sergei Romanenko && \large Peter Sestoft\\[0.1cm]
% Russian Academy of Sciences && Royal Veterinary and Agricultural University\\
% Moscow, Russia && Copenhagen, Denmark
% \end{tabular}
\end{center}

\vspace{1cm}

\subsection*{This document}

This manual describes the Moscow ML library, which includes parts of
the SML Basis Library and several extensions.  The manual has been
generated automatically from the commented signature files.


\subsection*{Alternative formats of this document}

\subsubsection*{Hypertext on the World-Wide Web}
  
The manual is available at
\verb$http://www.dina.kvl.dk/~sestoft/mosmllib/$ for online browsing.


\subsubsection*{Hypertext in the Moscow ML distribution}
  
The manual is available for offline browsing at
\verb$mosml/doc/mosmllib/index.html$ in the distribution.

\subsubsection*{On-line help in the Moscow ML interactive system}

The manual is available also in interactive {\tt mosml} sessions.
Type {\tt help "lib";} for an overview of built-in function libraries.
Type {\tt help "fromstring";} for help on a particular identifier,
such as {\tt fromString}.  This will produce a menu of all library
structures which contain the identifier {\tt fromstring} (disregarding
the lowercase/uppercase distinction):

{\small\begin{verbatim}
    --------------------------------
    |   1 | val  Bool.fromString   |
    |   2 | val  Char.fromString   |
    |   3 | val  Date.fromString   |
    |   4 | val  Int.fromString    |
    |   5 | val  Path.fromString   |
    |   6 | val  Real.fromString   |
    |   7 | val  String.fromString |
    |   8 | val  Time.fromString   |
    |   9 | val  Word.fromString   |
    |  10 | val  Word8.fromString  |
    --------------------------------
\end{verbatim}}

\noindent Choosing a number from this menu will invoke the help
browser on the desired structure, e.g.\ {\tt Int}.  

\vfill

\begin{center}
\begin{tabular}{|c|}\hline
\rule[-0.4cm]{0cm}{1cm}The Moscow ML home page is\ \
    \verb$http://www.dina.kvl.dk/~sestoft/mosml.html$\\\hline
\end{tabular}
\end{center}

\newpage

\setcounter{page}{1}

\twocolumn
{\setlength{\parindent}{0cm}
\renewcommand{\contentsline}[3]{#2\hfill #3\hspace*{3em}\newline}
\tableofcontents
}
\onecolumn

\newpage 

\input{texsigsigs.tex}

\newpage

\markboth{\MakeUppercase{Index}}{\MakeUppercase{Index}}
\addcontentsline{toc}{section}{Index}
% mosmllib.tex v. 2.00.0 Copyright (C) Peter Sestoft 2000, 2000-06-01
%
% You may edit for lay-out, or leave out irrelevant sections (if
% such omissions are marked somehow), but you may not redistribute the
% sources.  The authors' names and the Moscow ML URL must be left in place.

\documentclass[fleqn,twoside,a4paper]{article}
\usepackage{geometry}

\usepackage{isolatin1,program,pslatex}
\usepackage[T1]{fontenc}

\makeindex
\pagestyle{headings}
\thispagestyle{empty}

\begin{document}

\begin{center}

  \vspace*{0cm}

{\huge\bf Moscow ML Library Documentation}\\[0.5cm]

{Version 2.00 of June 2000}\\[0.5cm]

Sergei Romanenko, Russian Academy of Sciences, Moscow, Russia\\
Claudio Russo, Cambridge University, Cambridge, United Kingdom\\
Peter Sestoft, Royal Veterinary and Agricultural University, 
Copenhagen, Denmark

% \begin{tabular}{ccc}
% \large Sergei Romanenko && \large Peter Sestoft\\[0.1cm]
% Russian Academy of Sciences && Royal Veterinary and Agricultural University\\
% Moscow, Russia && Copenhagen, Denmark
% \end{tabular}
\end{center}

\vspace{1cm}

\subsection*{This document}

This manual describes the Moscow ML library, which includes parts of
the SML Basis Library and several extensions.  The manual has been
generated automatically from the commented signature files.


\subsection*{Alternative formats of this document}

\subsubsection*{Hypertext on the World-Wide Web}
  
The manual is available at
\verb$http://www.dina.kvl.dk/~sestoft/mosmllib/$ for online browsing.


\subsubsection*{Hypertext in the Moscow ML distribution}
  
The manual is available for offline browsing at
\verb$mosml/doc/mosmllib/index.html$ in the distribution.

\subsubsection*{On-line help in the Moscow ML interactive system}

The manual is available also in interactive {\tt mosml} sessions.
Type {\tt help "lib";} for an overview of built-in function libraries.
Type {\tt help "fromstring";} for help on a particular identifier,
such as {\tt fromString}.  This will produce a menu of all library
structures which contain the identifier {\tt fromstring} (disregarding
the lowercase/uppercase distinction):

{\small\begin{verbatim}
    --------------------------------
    |   1 | val  Bool.fromString   |
    |   2 | val  Char.fromString   |
    |   3 | val  Date.fromString   |
    |   4 | val  Int.fromString    |
    |   5 | val  Path.fromString   |
    |   6 | val  Real.fromString   |
    |   7 | val  String.fromString |
    |   8 | val  Time.fromString   |
    |   9 | val  Word.fromString   |
    |  10 | val  Word8.fromString  |
    --------------------------------
\end{verbatim}}

\noindent Choosing a number from this menu will invoke the help
browser on the desired structure, e.g.\ {\tt Int}.  

\vfill

\begin{center}
\begin{tabular}{|c|}\hline
\rule[-0.4cm]{0cm}{1cm}The Moscow ML home page is\ \
    \verb$http://www.dina.kvl.dk/~sestoft/mosml.html$\\\hline
\end{tabular}
\end{center}

\newpage

\setcounter{page}{1}

\twocolumn
{\setlength{\parindent}{0cm}
\renewcommand{\contentsline}[3]{#2\hfill #3\hspace*{3em}\newline}
\tableofcontents
}
\onecolumn

\newpage 

\input{texsigsigs.tex}

\newpage

\markboth{\MakeUppercase{Index}}{\MakeUppercase{Index}}
\addcontentsline{toc}{section}{Index}
\input{mosmllib.ind}

\end{document}

%%% Local Variables: 
%%% mode: latex
%%% TeX-master: t
%%% End: 


\end{document}

%%% Local Variables: 
%%% mode: latex
%%% TeX-master: t
%%% End: 


\end{document}

%%% Local Variables: 
%%% mode: latex
%%% TeX-master: t
%%% End: 


\end{document}

%%% Local Variables: 
%%% mode: latex
%%% TeX-master: t
%%% End: 
